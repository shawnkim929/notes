\documentclass[11pt,a4paper,english]{paper}
\usepackage{mathtools}
\usepackage[breakable]{tcolorbox}
%\usepackage{minted}
\newtcolorbox{mybox}[1]{colback=red!5!white,colframe=red!75!black,fonttitle=\bfseries,title=#1,breakable}
\newtcolorbox{bluebox}[1]{colback=blue!5!white,colframe=blue!75!black,fonttitle=\bfseries,title=#1,breakable}
\newtcolorbox{gbox}[1]{colback=green!5!white,colframe=green!75!black,fonttitle=\bfseries,title=#1,breakable}
\newtcolorbox{bbox}[1]{colback=black!5!white,colframe=black!75!black,fonttitle=\bfseries,title=#1,breakable}
\usepackage{amsmath}                                    % extensive math options
\usepackage{amssymb}                                    % special math symbols
\usepackage[mathlines]{lineno}
\usepackage[Gray,squaren,thinqspace,thinspace]{SIunits} % elegant units
\usepackage{listings}                                   % source code

\usepackage{graphicx}
\graphicspath{ {./} }
%\setminted{breaklines}

\begin{document}

\title{CS 008 \\ Lecture notes \\ 4/11/24}
\maketitle

\section{Outline}

\begin{itemize}

  \item Binary tree (recap)
  \item Data structures (Sequences vs Sets)
  \item Balanced (AVL) Trees (Subset of Binary Trees) (Skew \& Rotate)

\end{itemize}

\section{Binary Tree}

Traversal Order (In-order) goes from Left-Subtree $\rightarrow$ Parent $\rightarrow$ Right-Subtree.

\bigskip
\begin{bluebox}{Operations for Binary Trees:} {
\begin{itemize} {

\item first\_in\_subtree(node): find first node in the subtree defined by node
\item last\_in\_subtree(node)
\item successor(node): find the next node in traversal order after node
\item predecessor(node)
\item insert\_after(existing\_node, new\_node): insert new\_node in traversal order
\item insert\_before(existing\_node, new\_node)
\item delete(node): remove node from the tree while maintaining traversal order

}\end{itemize}
}\end{bluebox}

\bigskip
\noindent Details on implementation would've been added but they've already been mentioned multiple times before in lectures and can be googled at this point if it's not intuitive.

\section{Data structure theory}

Insert table


\bigskip
\noindent Notice how with some functions of certain implementations, we see a 1(a) time complexity. The $a$ stands for the amortized (aka: average) time because for example, if we had a dynamic array, there is a chance we need to extend the dynamic array.

\bigskip
\subsection{Set Binary Tree (BST)} 

One property of sets is the ability to also have keys. A binary search tree is considered a 'keyed' set where it is sorted by the key / value to determine it's position in the tree. Using a binary search tree, you can implement an entire set interface by just adding a key.

\subsection{Sequence Binary Tree}

Similarly to how a BST can be used to implement a set, the traversal order of the binary tree is also the sequence order. In summary, for BST Trees, we can implement a sequence or a set with the BST Tree because of the properties the BST Tree has that can be used for these purposes. Using specific properties for a BST Tree, we can implement certain abstract data types. 

\section{AVL Trees}

AVL Trees are a specific type of binary trees and have a specific property where it "self-balances" itself. Sometimes, trees will have an issue where 


\end{document}
