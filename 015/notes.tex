\documentclass[11pt,a4paper,english]{paper}
\usepackage{mathtools}
\usepackage[breakable]{tcolorbox}
%\usepackage{minted}
\newtcolorbox{mybox}[1]{colback=red!5!white,colframe=red!75!black,fonttitle=\bfseries,title=#1,breakable}
\newtcolorbox{bluebox}[1]{colback=blue!5!white,colframe=blue!75!black,fonttitle=\bfseries,title=#1,breakable}
\newtcolorbox{gbox}[1]{colback=green!5!white,colframe=green!75!black,fonttitle=\bfseries,title=#1,breakable}
\newtcolorbox{bbox}[1]{colback=black!5!white,colframe=black!75!black,fonttitle=\bfseries,title=#1,breakable}
\usepackage{amsmath}                                    % extensive math options
\usepackage{amssymb}                                    % special math symbols
\usepackage[mathlines]{lineno}
\usepackage[Gray,squaren,thinqspace,thinspace]{SIunits} % elegant units
\usepackage{listings}                                   % source code

\usepackage{graphicx}
\graphicspath{ {./} }
%\setminted{breaklines}

\begin{document}

\title{CS 008 \\ Lecture notes \\ 4/25/24}
\maketitle

\section{Direct Access Arrays and Hash Tables}

\subsection{Outline}

\begin{itemize}

  \item Pre-Lecture questions

\end{itemize}


\section{Pre-lecture questions}

Important note with AVT Trees:
\noindent 

\bigskip
\noindent \textbf{Big lecture question:} Is it possible to execute the function $find(k)$ any more quickly than $O(log(n))$?

\section{Word RAM model}

Any region of memory can be accessed in $O(1)$ complexity time. In reality it is not but for the most part this statement is true. In the word RAM model, anything that is within 64-bits is within $O(1)$. When we work with a data structure with $n$ elements, $n < 2^{w}$. For most computers, $w = 64$ and we call this a \textbf{word}. Memory is divided into w-bit chunks and each chunk can be read and written in $O(1)$.

\section{Comparison model of computation}

The comparison model of computation is more restrictive than the Word RAM Model. Like what the name implies, the comparison model can only perform comparisons (==, !=, \lesser, \greater, \leq, \geq)

\section{Direct access arrays}

Going back to our lecture question: Finding an 


\end{document}
