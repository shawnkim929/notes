\documentclass[11pt,a4paper,english]{paper}
\usepackage{mathtools}
\usepackage[breakable]{tcolorbox}
\usepackage{minted}
\newtcolorbox{mybox}[1]{colback=red!5!white,colframe=red!75!black,fonttitle=\bfseries,title=#1,breakable}
\newtcolorbox{bluebox}[1]{colback=blue!5!white,colframe=blue!75!black,fonttitle=\bfseries,title=#1,breakable}
\newtcolorbox{gbox}[1]{colback=green!5!white,colframe=green!75!black,fonttitle=\bfseries,title=#1,breakable}
\newtcolorbox{bbox}[1]{colback=black!5!white,colframe=black!75!black,fonttitle=\bfseries,title=#1,breakable}
\usepackage{amsmath}                                    % extensive math options
\usepackage{amssymb}                                    % special math symbols
\usepackage[Gray,squaren,thinqspace,thinspace]{SIunits} % elegant units
\usepackage{listings}                                   % source code

\setminted{breaklines}

\begin{document}

\title{CS 008 \\ Lecture notes \\ 3/14/24}
\maketitle

\section{Outline}
\begin{itemize}
  \item Announcement
  \item Templates
  \item Linked-lists and Double Linked-lists
  \item Big-O discussion
  \item Assignment 2 (Sequences part 2)

\end{itemize}

\section{Announcement}
\begin{gbox}{Announcement:}{

  There will be Office Hours held from Monday, Wednesday and Friday from 4-5 pm on Zoom. You can find the link on the Canvas homepage.

}\end{gbox}

\section{Templates}

When you have multiple classes that perform the same task but for different datatypes, you can create a template class instead. You can declare a template with \mintinline{cpp}{template <class Template_parameter_name>} on the line before the class definition of the header file.

\bigskip

\noindent When converted to a template class, various changes to member and non-member functions will be needed. Any functions that use the template class will need modifications. The implementation of templated class methods must be in the header file. Instead, put a include statement at the bottom of the file.

\bigskip

\begin{bluebox}{Note:} {

    See Section 6.2 of the textbook (Main and Savitch) for full syntax. \textbf{(Particularly the textbox on Page 304).}


}\end{bluebox}

\bigskip
\noindent Some examples of templates can be seen in the figures below.
\bigskip

\begin{bbox}{Figure: bag\_templated.h}{

  \begin{minted}{cpp}

#ifndef MAIN_SAVITCH_BAG5_H
#define MAIN_SAVITCH_BAG5_H

#include <cstdlib> // Provides NULL and size_t and NULL
#include "node2.h" // Provides node class

template <class Item>

class bag
{
    public:
        // TYPEDEFS
        typedef std::size_t size_type;
        typedef Item value_type;
        // CONSTRUCTORS and DESTRUCTOR
        bag( );
        bag(const bag& source);
        ~bag( );
        // MODIFICATION MEMBER FUNCTIONS
        size_type erase(const Item& target);
        bool erase_one(const Item& target);
        void insert(const Item& entry);
        void operator +=(const bag& addend);
        void operator =(const bag& source);
        // CONST MEMBER FUNCTIONS
        size_type count(const Item& target) const;
        Item grab( ) const;
        size_type size( ) const { return many_nodes; }

    private:
        node<Item> *head_ptr; // Head pointer for the list of items
        size_type many_nodes; // Number of nodes on the list
};

// NONMEMBER FUNCTIONS for the bag<Item> template class
template <class Item>
bag<Item> operator +(const bag<Item>& b1, const bag<Item>& b2);

// The implementation of a template class must be included in its header file:
#include "bag_templated.template"

#endif
\end{minted}

}\end{bbox}

\section{Linked lists and Doubly-Linked lists}

A linked list is a type of data structure that is comprised of list of nodes. Each node, except for the \textbf{head}, points to another node regardless of where it is in memory. The nodes contain data and a pointer to the next node.

\bigskip


\noindent The fundamental unit of both types is called a \textbf{node}. A node for a \textbf{linked list} consists of data and a pointer which points to the next node in the list. For a \textbf{doubly-linked list}, the nodes consist of data, a pointer which points to the next node in the list and the pointer which points to the previous node in the list.

\bigskip

\begin{bluebox}{Example:} {

    For a \textbf{linked list}, an example node can be seen here:

\begin{minted}{cpp}
struct node {

  node* next;
  (data type) data;

};
\end{minted}

\bigskip

    Both \textbf{linked lists} and \textbf{doubly-linked lists} use nodes. However, both nodes are different in small ways. An example of a doubly-linked node can be:

\begin{minted}{cpp}
struct node {

  node* next;
  node* previous;
  (data type) data;

}
\end{minted}
}
\end{bluebox}

\bigskip

\begin{gbox}{Question:} {

    How would you know where to find the last node in a linked list?

    \bigskip
    \textbf{Answer}: To get the last node in a linked list, you would need to iterate through the linked list and find a node that points to \textit{null}. Therefore, the node that points to \textit{null} will be the last.

    \bigskip
    What is the time complexity (O(..)) of finding the last node in a linked list?

    \bigskip
    \textbf{Answer}: The time complexity of finding the last node in a linked list is $O(n)$ where $n$ represents the number of nodes.


} \end{gbox}


\section{O(n) discussion}

Consider a function: $f(x) = a_{n}x^n + x^{n-1}$

\bigskip
\noindent \textbf{What would be the time complexity of this function?}
\bigskip

\noindent To figure out the time complexity of the function given, we must look for the highest order of the function: $a_{n}x^n$. The coefficient is then dropped to reveal $x^n$ which is now the function's time complexity, $O(x^n)$.

\bigskip

\begin{gbox}{Question:} {

  If we had an array of integers with a size of 10 and we have 5 elements filled up in the array, what would be the time complexity of adding a 6th element to the array?

  \bigskip
  \textbf{Answer}: It would be $O(1)$ since it would take 2 operations: Getting the 6th index in the array and setting a value for the 6th element. The coefficient is dropped from the largest power and is reduced to 1. Therefore, the result is $O(1)$.


  }
\end{gbox}

\section{Sequence part 2}

There will be a new assignment that will revisit Sequences and also templates.

\bigskip

\noindent The new assignment is designed to:
\begin{itemize} 
  \item Get you more familiar with templates
  \item To understand the emphasis on the header (.h) file and the implementation (.cpp)
  \item Get more comfortable analyzing the run time complexity of real functions.
\end{itemize}

\bigskip

\end{document}
